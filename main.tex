\documentclass[10pt,a5paper]{article}
%\documentclass[10pt,a4paper]{article}
\usepackage[utf8]{inputenc}

%opcionális:
\usepackage[margin=0.5in]{geometry}
%\usepackage[margin=1in]{geometry}

\usepackage{systeme}
\usepackage[pdftex]{graphicx}
\usepackage{hyperref}
\usepackage{amsmath}
\usepackage{amsfonts}
\usepackage{amssymb}
\usepackage{indentfirst}
%\author{Apagyi Dávid \\ \normalsize{apagyi.david@gmail.com}}
\title{Fizika alapfogalmak}
\date{2022. június 2.}


% \DeclareMathSizes{16}{30}{16}{16}


\newcounter{alapfogalom}%[section]
\newenvironment{af}[1][]{\refstepcounter{alapfogalom}\par\medskip
   \noindent \textbf{\thealapfogalom. #1:} \addcontentsline{toc}{subsection}{{#1}} \rmfamily}{\smallskip}

\usepackage[many]{tcolorbox}
\usetikzlibrary{decorations.pathreplacing}
\newtcolorbox{leftbrace}{%
    enhanced jigsaw, 
    breakable, % allow page breaks
    frame hidden, % hide the default frame
    overlay={%
        \draw [
            decoration={brace,amplitude=0.5em},
            decorate,
            ultra thick,
        ]
        % right line
        (frame.north west)--(frame.south west);
    },
    % paragraph skips obeyed within tcolorbox
    parbox=false,
}

\renewcommand\labelitemi{-}

\begin{document}
%\newpage $\;$
\maketitle

\renewcommand*\contentsname{Tartalomjegyzék}
\tableofcontents
\newpage

\section{Bevezetés}

\begin{af}[Természet]
A létező anyagi világ.
\end{af}
\begin{af}[Természettudományok]
A természetre vonatkozó ismeretek rendszere.
\end{af}
\begin{af}[Természettudományok célja]
A jelenségek törvényszerűségeinek felismerése és alkalmazása, az emberi tevékenység előremozdítása, valamint új jelenségek előrejelzése.
\end{af}
\begin{af}[Fizika]
Az anyag általános tulajdonságait, törvényeit vizsgálja.
\end{af}
\begin{af}[Megfigyelés]
Tárgyak, jelenségek, folyamatok jellemzőinek spontán nyomonkövetése.
\end{af}
\begin{af}[Kísérlet]
Egy jelenség előzetes terv alapján történő szándékos előidézése, és pontos megfigyelése, egy alkalommal egy dologra összpontosítva.
\end{af}
\begin{af}[Modellalkotás]
Egy kép megalkotása, amellyel az anyagok viselkedését megmagyarázzuk úgy, hogy már egy ismert dologhoz hasonlítunk.
\end{af}
\begin{af}[Mérés]
Méréskor azt állapítjuk meg, hogy a mérendő mennyiség hányszorosa az egységül választottnak.
\end{af}
\begin{af}[Kinematika]
A kinematika leírja a mozgásokat.
\end{af}
\begin{af}[Dinamika]
A dinamika a mozgások okait vizsgálja.
\end{af}
\begin{af}[Energetika]
Az energetika a mozgásokkal kapcsolatos energiákkal foglalkozik.
\end{af}



\newpage
\section{Kinematika}

\begin{af}[Pálya]
Az a vonal, amelyen a test a mozgása közben végighalad.
\end{af}
\begin{af}[Út]
A pálya azon részének hossza, amelyen a mozgást vizsgáljuk.
\end{af}
\begin{af}[Elmozdulás]
$\Delta r$, a helyvektor megváltozása.
\end{af}
\begin{af}[A mechanika feladata]
A mechanika feladata a hely megadása idő függvényében.
\end{af}
\begin{af}[Sebesség]
\begin{itemize}
    \item $v=\dfrac{\Delta s}{\Delta t}$
    \item $\left[v\right]=\dfrac{m}{s}$
    \item Számértéke megmutatja az egységnyi idő alatt befutott elmozdulást.
\end{itemize}
\end{af}
\begin{af}[Az EVE mozgás kritériumai] %$\vec{v}=\text{áll.}$, azaz:
\begin{itemize}
    \item pálya: egyenes
    \item $v=\text{áll.}$: egyenletes
\end{itemize}
\end{af}
\begin{af}[Sebességvektor]
\begin{itemize}
    \item $\vec{v}=\dfrac{\Delta\vec{r}}{\Delta t}$
    \item $\left[|\vec{v}|\right]=\dfrac{m}{s}$
    \item Megmutatja az egységnyi idő alatt bekövetkező elmozdulást.
    \item Iránya: elmozdulás irányú.
\end{itemize}
\end{af}
\begin{af}[Gyorsulás]
\begin{itemize}
    \item $a=\dfrac{\Delta v}{\Delta t}$
    \item $\left[a\right]=\dfrac{m}{s^2}$
    \item Számértéke megmutatja az egységnyi idő alatt bekövetkező sebességváltozást.
\end{itemize}
\end{af}
\begin{af}[Az EVEV mozgás kritériumai] %$\vec{v}=\text{áll.}$, azaz:
\begin{itemize}
    \item pálya: egyenes
    \item $a=\text{áll.}$: egyenletesen változó
\end{itemize}
\end{af}
\begin{af}[Gyorsulásvektor]
\begin{itemize}
    \item $\vec{a}=\dfrac{\Delta\vec{v}}{\Delta t}$
    \item $\left[|\vec{a}|\right]=\dfrac{m}{s^2}$
    \item Megmutatja az időegység alatt bekövetkező sebességváltozás vektort.
    \item Iránya: sebességváltozás irányú.
\end{itemize}
\end{af}
\begin{af}[Szabadesés]
\begin{itemize}
    \item Kritériumai:
    \begin{itemize}
        \item $v_0=0$
        \item $a=g$
    \end{itemize}
    \item Számolás:
    \begin{itemize}
        \item $\Delta y=\dfrac{g}{2}\Delta t^2$
        \item $v=g\Delta t$
    \end{itemize}
\end{itemize}
\end{af}
\begin{af}[Hajítás lefelé]
\begin{itemize}
    \item Kritériumai:
    \begin{itemize}
        \item $v_0\not=0$
        \item $a=g$
    \end{itemize}
    \item Számolás:
    \begin{itemize}
        \item $\Delta y=v_0\Delta t+\dfrac{g}{2}\Delta t^2$
        \item $v=v_0+g\Delta t$
    \end{itemize}
\end{itemize}
\end{af}
\begin{af}[Hajítás felfelé]
\begin{itemize}
    \item Kritériumai:
    \begin{itemize}
        \item $v_0\not=0$
        \item $a=-g$
    \end{itemize}
    \item Számolás:
    \begin{itemize}
        \item $\Delta y=v_0\Delta t-\dfrac{g}{2}\Delta t^2$
        \item $v=v_0-g\Delta t$
    \end{itemize}
\end{itemize}
\end{af}
\begin{af}[Periodikus mozgás] Periodikos mozgásról beszélünk, ha a test ugyanazt a mozgásszakaszt ugyanúgy ismételgeti.
\end{af}
\begin{af}[Anyagi pont]
Anyagi pontról beszélünk, ha a test ún. tiszta haladó mozgást végez, vagy méretei elhanyagolhatóak az elmozduláshoz, illetve a közte és más testek közötti távolsághoz képest.
\end{af}
\begin{af}[Periódusidő]
\begin{itemize}
    \item $T=\dfrac{\Delta t}{N}$
    \item %mértékegysége:
    $\left[T\right]=s$
    \item Megmutatja egy periódus vagy körülfordulás megtételéhez szükséges időt.
\end{itemize}
\end{af}
\begin{af}[Fordulatszám]
\begin{itemize}
    \item $n=\dfrac{N}{\Delta t}$
    \item %mértékegysége:
    $\left[n\right]=\dfrac{1}{s}$
    \item Megmutatja egy egységnyi idő alatt bekövetkező körülfordulások számát.
\end{itemize}
\end{af}
\begin{af}[Szögsebesség]
\begin{itemize}
    \item kiszámítás: $\omega=\dfrac{\Delta \varphi}{\Delta t}$
    \item %mértékegysége:
    $\left[\omega\right]=\dfrac{(rad)}{s}=\dfrac{1}{s}$
    \item Megmutatja egy egységnyi idő alatt bekövetkező szögelfordulást.
\end{itemize}
\end{af}
\begin{af}[Mozgás]
Mozgásról beszélünk, ha az anyagi pont helye megváltozik a választott vonatkoztatási rendszerben.
\end{af}



\newpage
\section{Anyagi pont dinamikája}
\begin{af}[Tehetetlenség törvénye (Newton I.)]
\begin{itemize}
    \item[a)] Minden test megtartja nyugalmi állapotát vagy EVE mozgását mindaddig, míg más testek vagy mezők ennek megváltoztatására nem kényszerítik.
    \item[b)] MÁV mindig KH eredményeként jön létre.
    \item[c)] KH hiányában a testek EVE mozgást végeznek.
\end{itemize}
\end{af}
\begin{af}[Inerciarendszer]
Olyan vonatkoztatási rendszer, melyben érvényes Newton I. törvénye.
\end{af}
\begin{af}[Galilei-féle relativitási elv]
A Földhöz vagy bármilyen más IR-hez képest állandó sebességgel haladó vonatkoztatási rendszer is IR.
\end{af}
\begin{af}[Rugalmas kölcsönhatás]
A kölcsönhatás megszünte utána a testek visszanyerik eredeti állapotukat.
\end{af}
\begin{af}[Tömeg] A tehetetlenség mértéke.
\begin{itemize}
    \item jele: $m$
    \item $\left[m\right]=kg$
\end{itemize}
\end{af}
\begin{af}[Dinamikai tömegmérés elve]
Egy test tömege akkor $N$-szeres egy másik test tömegéhez viszonyítva ha vele párkölcsönhatásba hozva a sebességváltozása $\dfrac{1}{N}$-szeres.
\end{af}
\begin{af}[Lendület(vektor)]
\begin{itemize}
    \item $\vec{I}=m\vec{v}$
    \item mértékegysége: $\left[I\right]=kg\dfrac{m}{s}$
\end{itemize}
\end{af}
\begin{af}[Zárt (test)rendszer] Zárt testrendszerről beszélünk, ha a testek környezettel való kölcsönhatásaitól eltekinthetünk.
\end{af}
\begin{af}[Lendületmegmaradás törvénye]
\begin{itemize}
    \item[] LMT$_1$: Zárt testrendszerben a testek lendületváltozásainak vektori összege nullvektor.
    $$\sum_{i=1}^n\Delta I_i=\vec{0}$$
    \item[] LMT$_2$: Zárt testrendszerben a testek lendületének vektori összege állandó.
    $$\sum_{i=1}^n\ I_i=\text{áll.}$$
\end{itemize}

\end{af}
\begin{af}[Erő(vektor)]
\begin{itemize}
    \item $\vec{F}=\dfrac{ I}{\Delta t}$
    \item $\left[F\right]=\dfrac{kg\frac{m}{s}}{s}=kg\dfrac{m}{s^2}=N$
    \item Megmutatja, az egységnyi idő alatt bekövetkező lendületváltozást.
    \item Iránya lendületváltozás irányú.
\end{itemize}
\end{af}
\begin{af}[Támadáspont]
A test azon pontja, ahol az erőhatás éri. (kiterjedt test)
\end{af}
\begin{af}[Hatásvonal]
Az erő támadáspontján átfektetett, az erővel párhuzamos egyenes.
\end{af}
\begin{af}[Newton II. törvénye]
Ha egy $m$ tömegű testre $\vec{F}$ erő hat, akkor az a test $\vec{a}=\dfrac{\vec{F}}{m}$ gyorsulással fog mozogni.
$$\vec{a}=\dfrac{\vec{F}}{m}$$
\end{af}
\begin{af}[Newton III. törvénye]
(Hatás-ellenhatás törvénye, kölcsönhatás-törvény) \\
Az egy KH.-ban fellépő erők azonos nagyságúak, ellentétes irányításúak, közös hatásvonalúak és különböző testekre hatnak.
$$\vec{F}_{2\xrightarrow{}1}=-\vec{F}_{1\xrightarrow{}2}$$
\end{af}
\begin{af}[Newton IV. törvénye]
Az egy testre ható erők összegezhetők vektorilag, és helyettesíthetők egy ún. eredő erővel.
$$\sum_{i=1}^n\vec{F_i}=\vec{F_e}$$
\end{af}
\begin{af}[Súlyerő]
Az az erő, ammelyel a test húzza a felfüggesztést és/vagy nyomja az alátámasztást.
\end{af}
\begin{af}[Erőtörvény]
Az erőhatást kifejtő környezet és a test jellemzőivel megadott matematikai kifejezés, amellyel az erő nagyságát és irányát adhatjuk meg.
\end{af}
\begin{af}[Rugóállandó]
\begin{itemize}
    \item $D=\dfrac{F_r}{\Delta l}$
    \item $\left[D\right]=\dfrac{N}{m}$
    \item Megmutatja, hogy egységnyi megnyúlás esetén mekkore erőt fejt kia rugó.
\end{itemize}
\end{af}
\begin{af}[Tapadási erő]
Az egymáshoz képest nyugvó felületek által egymásra kifejtett erő érintőirányú összetevője.
\end{af}
\begin{af}[Súrlódási erő]
Az egymáshoz képest mozgó felületek által egymásra kifejtett erő sebességgel ellentétes irányú összetevője.
\end{af}
\begin{af}[Kényszererő]
Kényszererő esetén, mivel az erőkifejtés elhanyagolható deformációval jár, a pályaalak előre meghatározott.
\end{af}
\begin{af}[Geostacionárius műholdak]
Olyan mesterséges holdak, melyek mindig a Föld ugyanazon pontja felett tartózkodnak.
\end{af}



\newpage
\section{Pontrendszerek dinamikája}
\begin{af}[Pontrendszer]
Egymással kölcsönhatásban lévő pontszerű testek rendszere.
\end{af}
\begin{af}[Külső erők]
A rendszerhez nem tartozó testek fejtik ki.
\end{af}
\begin{af}[Belső erők]
A rendszer tagjai fejtik ki egymásra.
\end{af}
\begin{af}[Zárt pontrendszer]
Zárt pontrendszerről beszélünk, ha a külső erők eredője zérus.
\end{af}
\begin{af}[Tömegközéppont]
Minden pontrendszernek (és kiterjedt testnek) van legalább egy olyan pontja, amely kh. hiányában EVE mozgástvégez, ez a kitüntetett pont a tömegközéppont.
\end{af}
\begin{af}[Tömegközéppont-tétel]
Pontrendszer tömegközéppontja úgy mozog, mintha benne a pontrendszer teljes tömege egyesítve volna, és rá a külső erők eredője hatna.
$$m_{ö}a_{xT}=\sum F_{ixk}$$
\end{af}
\begin{af}[Merev test]
Olyan pontrendszer, amelyben a részecskék egymáshoz viszonyított távolsága és helyzete nem változik.
\end{af}
\begin{af}[Forgatónyomaték]
\begin{itemize}
    \item Az erőhatást jellezmi forgatóhatás szempontjából.
    \item $M=Fk$
\end{itemize}
\end{af}
\begin{af}[Erőkar]
Az erő hatásvonalának forgástengelytől mért távolsága.
\end{af}
\begin{af}[Erőpár]
Az ugyanarra a merev testre ható két erőt, amelyek ellentétes irányúak, párhuzamos hatásvonalúak és egyenlő nagyságúak, erőpárnak nevezzük. Az erőpár nem helyettesíthető egyetlen erővel.
\end{af}
\begin{af}[Erőpár forgatónyomatéka]
Erőpár forgatónyomatéka egyenlő az egyik erő nagyságának és a két erő hatásvonala közötti távolságnak a szorzatával, forgástengelytől függetlenül.
$$M=Fd$$
\end{af}



\newpage
\section{Tömegpont mozgásának energetikai leírása}
\begin{af}[Munkavégzés]
Munkavégzésről beszlünk, ha erőhatás következtében elmozdulás jön létre.
\begin{itemize}
    \item mértéke: munka
    \item jele: $W$
    \item $\left[W\right]=J$
\end{itemize}
\end{af}
\begin{af}[Energia]
Testek, mezők változást okozó képessége.
\begin{itemize}
    \item jele: $E$
    \item $\left[E\right]=J$
    \item állapotot jellemez
    \item skaláris mennyiség
    \item viszonylagos mennyiség
    \item kh. közben változik
    \item megmaradási törvény érvényes rá
    \item kvantumos (adagos)
\end{itemize}
\end{af}
\begin{af}[Energiatartalom mértéke]
Annak a munkavégzésnek a mértéke, amellyel az adott állapot kialakítható egy önkényesen kiválasztott alapállapotból kiindulva.
\end{af}
\begin{af}[Munkatétel]
Pontszerű test mozgási energiájának megváltozása megegyezik a testre ható erők eredőjének munkájával (avagy az erők munkájának összegével).
$$\Delta E_m=W_e$$
\end{af}
\begin{af}[Konzervatív erő (kh.)]
Az az erő (kh.), melynek munkája függetlena befutott úttól, csak a kezdő és a végpont helyzete a meghatározó.
\end{af}
\begin{af}[Mechanikai energiamegmaradás törvénye]
Egy testrendszer mechanikai energiája állandó, ha tagjai között csak konzervatív kölcsönhatások jelennek meg.
$$\sum E_{mech}=\text{áll.}$$
\end{af}
\begin{af}[Teljesítmény]
\begin{itemize}
    \item $P=\dfrac{W}{\Delta t}$
    \item $\left[P\right]=\dfrac{J}{s}=W$
    \item Megmutatja az egységnyi idő alatt végzett munkát.
\end{itemize}
\end{af}
\begin{af}[Egyszerű gépek]
Olyan eszközök, melyek az általuk kifejtett erő nagyságát, irányát és támadáspontját számunkra kedvező módon befolyásolják.
\end{af}
\begin{af}[Emelő]
Olyan merev rúd, amely velamely pontja, mint tengely körül elfordulhat.
\end{af}
\begin{af}[Csigák]
Olyan korongok, melyek peremén bemélyedés található és valamely pontjuk, mint tengely körül elfordulnak.
\end{af}
\begin{af}[Kepler I. törvénye]
A bolygók ellipszis alakú pályán keringenek, amelyeknek egyik fókuszpontjában a Nap áll.
\end{af}
\begin{af}[Kepler II. törvénye]
A Naptól a bolygóhoz húzott vezérsugár egyenlő időközök alatt egyenlő területeket súrol.
\end{af}
\begin{af}[Kepler III. törvénye]
A bolygók keringési időinek négyzetei úgy aránylanak egymáshoz, mint a bolygópályák fél nagytengelyeinek (pályasugarainak) köbei.
$$\dfrac{T_1^2}{T_2^2}=\dfrac{a_1^3}{a_2^3}$$
\end{af}



\newpage
\section{Folyadékok mechanikája}
\begin{af}[Folyadékok modellje]
Nagyszámú, apró, keményfalú golyók halmaza, melyek rendezetlen, gördülő mozgásukkal szorosan, de nem hézagmentesen töltik ki a teret. Közelítéskor taszító, távolításkor és alaphelyzetben rövid hatótávolságú vonzó kh. jelenik meg.
\end{af}
\begin{af}[Pascal törvénye]
A folyadék felszínére kifejtett erő által okozott nyomás a folyadékban minden helyen, minden irányban azonos mértékben jelenik meg.
\end{af}
\begin{af}[Nyomás]
\begin{itemize}
    \item $p=\dfrac{F}{A_\bot}$
    \item $\left[p\right]=\dfrac{N}{m^2}=Pa$
    \item Megmutatja az egységnyi erőre merőleges felületen megjelenő erőt.
\end{itemize}
\end{af}
\begin{af}[Hidrosztatikai nyomás]
A folyadék súlyából származó nyomás.
\end{af}
\begin{af}[Közlekedő edény]
Felül nyitott, alul csővel összekötött edények, melyekben a folyadék szabadon áramolhat.
\end{af}
\begin{af}[Légnyomás]
A levegő súlyából származó nyomás.
\end{af}
\begin{af}[Arkhimédesz törvénye]
A folyadékba merülő testre ható felhajtóerő nagysága megegyezik az általa kiszorított folyadék súlyával.
\end{af}
\begin{af}[Kohéziós erő (kh.)]
Valamely anyag azonos részecskéi között működő vonzóerő. (víz-víz hidrogénkötés, Hg-Hg fémkötés)
\end{af}
\begin{af}[Adhéziós erő (kh.)]
Egymással érintkező, különböző anyagok részecskéi között működő vonzóerő. (kh.) 
\end{af}
\begin{af}[Felületi feszültség (energetikai)]
\begin{itemize}
    \item $\alpha=\dfrac{\Delta E}{\Delta A}$
    \item $\left[\alpha\right]=\dfrac{J}{m^2}$
    \item Megmutatja, hogy a hártya felületének egységnyivel való növelése mekkore energiatöbbletet jelent.
\end{itemize}
\end{af}
\begin{af}[Felületi feszültség (erőtani)]
\begin{itemize}
    \item $\alpha=\dfrac{F_{fel}}{\Delta l}$
    \item $\left[\alpha\right]=\dfrac{N}{m}$
    \item Megmutatja a felületi réteg által a felületet határoló egységnyi hosszúságú vonaldarabra ható erő nagyságát.
\end{itemize}
\end{af}
\begin{af}[Folytonossági (vagy kontinuitási) egyenlet]
$$A_1v_1=A_2v_2$$
\end{af}
\begin{af}[Bernoulli-törvény]
$$p+\frac{1}{2}\varrho v^2+\varrho gh=\text{áll.}$$
\end{af}



\newpage
\section{Elektrosztatika}
\begin{af}[Vezetők]
Olyan anyagok, melyekben elmozdulásra képes töltéshordozók találhatóak.
\end{af}
\begin{af}[Vezetők csoportosítása]
\begin{enumerate}
    \item[a)] Elsőfajú vezetők: elektronok mozognak bennük.
    \item[b)] Másodfajú vezetők: ionok mozognak bennük.
\end{enumerate}
\end{af}
\begin{af}[Töltés mennyiségi jelentése]
Az elektromos állapot mennyiségi jellemzésére szolgál.
\end{af}
\begin{af}[1 C töltés]
$1\ C$ az a töltés, amely a vákuumban tőle $1\ m$ távolságra lévő szintén $1\ C$ töltésre $9\cdot10^9N$ erővel hat.
\end{af}
\begin{af}[Elektromos térerősség]
\begin{itemize}
    \item $\vec{E}=\dfrac{\vec{F}}{q}$
    \item $\left[E\right]=\dfrac{N}{C}$
    \item Megmutatja az egységnyi pozitív töltésre ható erőt.
    \item Iránya megegyezik a pozitív töltésre ható erő irányával.
\end{itemize}
\end{af}
\begin{af}[Elektromos erővonalak minőségi jelentése]
Az elektromos erővonalak olyan görbék, melyeknek bármely pontjához húzott érintő az ottani $\vec{E}$ tartóegyenese.
\end{af}
\begin{af}[Elektromos erővonalak mennyiségi jelentése]
Az erővonalképet úgy kell megrajzolni, hogy ahol a térerősség $E$ nagyságú, akkor ott a vonalakra merőlegesen felvett $A_\bot$ felületen $\Psi=EA_\bot$ számú erővonal haladjon keresztül.
\end{af}
\begin{af}[Feszültség]
\begin{itemize}
    \item $U_{AB}=\dfrac{W_{AB}}{q}$
    \item $\left[U_{AB}\right]=\dfrac{J}{C}=V$
    \item Megmutatja, mekkora munkát végez az elektromos tér, miközben az egységnyi pozitív töltéshordozó A pontból B pontba jut.
\end{itemize}
\end{af}
\begin{af}[Potenciál]
\begin{itemize}
    \item $U(P)=\dfrac{W_{P\infty}}{q}$
    \item $\left[U\right]=\dfrac{J}{C}=V$
    \item Megmutatja, mekkora munkát végez az elektromos tér, miközben az egységnyi pozitív töltéshordozó P pontból a végtelenbe jut.
\end{itemize}
\end{af}
\begin{af}[Elektromos megosztás]
Olyan fémes vezetőn lezajló jelenség, melynek során egy külső, úgynevezett megosztó töltés hatására a fém ellentétes oldalain ellentétes töltés jelenik meg. ($\sum Q=0$)
\end{af}
\begin{af}[Elektromos megosztás hatásai]
\begin{itemize}
    \item Az erővonalak a külső felületre merőlegesen helyezkednek el.
    \item A fém belsejében a térerősség zérus. ($\vec{E}=\vec{0}$)
    \item A töltések a külső felületen helyezkednek el.
    \item A fém minden pontja ekvipotenciális.
\end{itemize}
\end{af}
\begin{af}[Kapacitás]
\begin{itemize}
    \item $C=\dfrac{\Delta Q}{\Delta U}$
    \item $\left[C\right]=\dfrac{C}{V}=F$
    \item Megmutatja az egységnyi potenciálnöveléshez szükséges töblettöltés mértékét. 
\end{itemize}
\end{af}
\begin{af}[Eredő kapacitás]
Annak a kondenzátornak a kapacitása, amelyen ugyanakkora töltés hatására, ugyanakkora feszültség jelenik meg, mint az eredeti kapcsoláson.
\end{af}
\begin{af}[Párhuzamos kapcsolás törvényszerűségei]
\begin{itemize}
    \item Töltések összegződnek: $Q=Q_1+Q_2$
    \item Feszültségek megegyeznek: $U_1=U_2=U$
    \item Eredő kiszámítása: $C_e=\sum_{i=1}^n{C_i}$
\end{itemize}
\end{af}
\begin{af}[Soros kapcsolás törvényszerűségei]
\begin{itemize}
    \item Töltések megegyeznek: $Q_1=Q_2=Q$
    \item Feszültségek összegződnek: $U=U_1+U_2$
    \item Eredő kiszámítása: $\dfrac{1}{C_e}=\sum_{i=1}^n\dfrac{1}{C_i}$
\end{itemize}
\end{af}



\newpage
\section{Hőtani folyamatok}
\begin{af}[Lineáris hőtágulási együttható]
\begin{itemize}
    \item $\alpha=\dfrac{\Delta l}{l_0\Delta t}$
    \item $\left[\alpha\right]=\dfrac{m}{m^\circ C}=\dfrac{1}{^{\circ}C}$
    \item Megmutatja az egységnyi hosszúságú rúd hőmérsékletének egységnyivel való változtatása mekkora hosszúságváltozást eredményez.
\end{itemize}
\end{af}
\begin{af}[Térfogati hőtágulási együttható]
\begin{itemize}
    \item $\beta=\dfrac{\Delta V}{V_0\Delta t}$
    \item $\left[\beta\right]=\dfrac{1}{^{\circ}C}$
    \item Megmutatja az egységnyi térfogatú folyadék hőmérsékletének egységnyivel való változtatása mekkora térfogatváltozást eredményez.
\end{itemize}
\end{af}
\begin{af}[Gázok modellje]
Nagyszámú, apró, gyors mozgású golyók halmaza, melyek a teret rendezetlen mozgásukkal (röpködésükkel) lazán töltik ki, miközben ütköznek egymással és a tárolóedény falával.
\end{af}
\begin{af}[Izoterm folyamat]
\begin{itemize}
    \item $N=\text{áll.}$
    \item $t=\text{áll.}$
\end{itemize}
\end{af}
\begin{af}[Izobár folyamat]
\begin{itemize}
    \item $N=\text{áll.}$
    \item $p=\text{áll.}$
\end{itemize}
\end{af}
\begin{af}[Izochor folyamat]
\begin{itemize}
    \item $N=\text{áll.}$
    \item $V=\text{áll.}$
\end{itemize}
\end{af}
\begin{af}[Boyle-Mariotte-törvény]
Állandó mennyiségű ideális gáz izotermikus állapotváltozása során a gáz nyomásának és térfogatának szorzata állandó.
$$pV=\text{áll.}$$
\end{af}
\begin{af}[Abszolút hőmérsékleti skála]
\begin{itemize}
    \item Alappontja a hőmérséklet elvi alsó határa.
    \item Az egységek megegyeznek a Celsius-skála egységeivel.
\end{itemize}
\end{af}
\begin{af}[Gay-Lussac I.]
Állandó mennyiségű ideális gáz izobár állapotváltozása során a gáz térfogata és abszolút hőmérséklete egymással egyenesen arányos.
$$\dfrac{V}{T}=\text{áll.}$$
\end{af}
\begin{af}[Gay-Lussac II.]
Állandó mennyiségű ideális gáz izochor állapotváltozása során a gáz nyomása és abszolút hőmérséklete egymással egyenesen arányos.
$$\dfrac{p}{T}=\text{áll.}$$
\end{af}
\begin{af}[Gáztörvény]
A gáz kettő vagy több állapotát leíró állapotjelzők kapcsolata.
\end{af}
\begin{af}[Általános gáztörvény]
Állandó mennyiségű ideális gáz tetszőleges állapotváltozása során a nyomás és térfogat szorzata egyenesen arányos a gáz abszolút hőmérsékletével.
$$\dfrac{pV}{T}=\text{áll.}$$
\end{af}
\begin{af}[Állapotegyenlet]
A gáz egy adott állapotát leíró paraméterek kapcsolata.
\end{af}
\begin{af}[Belső energia]
A halmazt alkotó részecskék energiáinak összege.
$$E_b=\sum\varepsilon_{i\:mozg}+\sum\varepsilon_{i\:kh}$$
\end{af}
\begin{af}[Hőközlés]
\begin{itemize}
    \item Rendezetlen úton történő energiaközlés.
    \item jele: $Q$
\end{itemize}
\end{af}
\begin{af}[Munkavégzés]
\begin{itemize}
    \item Rendezett úton történő energiaközlés.
    \item jele: $W$
\end{itemize}
\end{af}
\begin{af}[Hőtan I. főtétele]
Egy halmaz belső energiája termikus és/vagy mechanikai kölcsönhatással változtatható meg. Mértéke a közölt hő és/vagy a környezet munkájának összegével egyezik meg.
$$\Delta E_b=Q+W_k$$
\end{af}
\begin{af}[Izotermikus folyamatok hőtani jellemzése]
\begin{itemize}
    \item $Q=W_{gá}$
    \item A gázzal közölt hő teljes egészében tágulási munkára fordítódik.
\end{itemize}
\end{af}
\begin{af}[Izobár folyamatok hőtani jellemzése]
\begin{itemize}
    \item $Q=\Delta E+W_t$
    \item A közölt hő egy része növeli a belső energiát, egy másik része pedig tágulási munkára fordítódik.
\end{itemize}
\end{af}
\begin{af}[Izochor folyamatok termodinamikai jellemzése]
\begin{itemize}
    \item $Q=\Delta E$
    \item A közölt hő teljes egészében a belső energiát növeli.
\end{itemize}
\end{af}
\begin{af}[Adiabatikus folyamatok termodinamikai jellemzése]
\begin{itemize}
    \item $\Delta E=W_k$
    \item A környezet munkája teljes egészében növeli a belső energiát.
\end{itemize}
\end{af}
\begin{af}[Hőkapacitás]
\begin{itemize}
    \item $C=\dfrac{Q}{\Delta T}$
    \item $\left[C\right]=\dfrac{J}{K}$
    \item Megmutatja az egységnyi hőmérsékletváltozás létrehozásához szükséges hőközlés mértékét.
\end{itemize}
\end{af}
\begin{af}[Fajhő]
\begin{itemize}
    \item $c=\dfrac{C}{m}=\dfrac{\frac{Q}{\Delta T}}{m}=\dfrac{Q}{m\Delta T}$
    \item $\left[c\right]=\dfrac{J}{kgK}$
    \item Megmutatja az egységnyi tömegű halmaz hőmérsékletének egységnyivel való változtatása mekkora hőközlést igényel.
\end{itemize}
\end{af}
\begin{af}[Mólhő]
\begin{itemize}
    \item $c^*=\dfrac{C}{n}=\dfrac{\frac{Q}{\Delta T}}{n}=\dfrac{Q}{n\Delta T}$
    \item $\left[c^*\right]=\dfrac{J}{molK}$
    \item Megmutatja az egységnyi anyagmennyiségű halmaz hőmérsékletének egységnyivel való változtatása mekkora hőközlést igényel.
\end{itemize}
\end{af}
\begin{af}[Szabadsági fok]
\begin{itemize}
    \item Az energiatárolás független lehetőségeinek száma. (Ahány négyzetes tag szerepel az energia kifejezésében.)
    \item jele: $f$
\end{itemize}
\end{af}
\begin{af}[Ekvipartíció tétele]
$T$ egyensúlyi hőmérsékletű halmazban minden részecske minden szabadsági fokára $\dfrac{1}{2}kT$ energia jut.
$$\varepsilon_x=\dfrac{1}{2}kT$$
\end{af}
\begin{af}[Hőtan II. főtétele]
Zárt anyaghalmazban önmagától olyan változások mennek végbe, melyeknek során az egy szabadsági fokra jutó energiák kiegyenlítődnek.\\
(\textit{A halmaz rendetlenebbé válik.}) \\
(\textit{Nő az entrópiája.})
\end{af}
\begin{af}[Kristályos, szilárd anyagok modellje]
Nagyszámú, apró, gyorsmozgású golyók halmaza, melyek a teret szabályos rendben, szorosan töltik ki, miközben helyhez kötött rezgőmozgást végeznek. Közelítéskor erős taszító, távolításkor és alaphelyzetben rövid hatótávolságú, erős vonzó kölcsönhatás jelenik meg.
\end{af}
\begin{af}[Olvadás]
Olvadásról beszélünk, ha a kristályos anyag folyamatos energiaközlés hatására egy jól
meghatározott állandósult hőmérsékleten (olvadáspont) folyadék halmazállapotúvá válik. Ekkor a közölt hő nem a hőmozgásra, hanem a részecskék közötti kölcsönhatások fellazítására fordítódik.
\end{af}
% \begin{af}[Olvadáspont]
% Az a hőmérséklet ($t_o$), amelyen az olvadás végbemegy.
% \end{af}
\begin{af}[Olvadáshő]
\begin{itemize}
    \item $L_o=\dfrac{Q}{m}$
    \item $\left[L_o\right]=\dfrac{J}{kg}$
    \item Megmutatja az előzőleg olvadáspontjára felmelegített egységnyi tömegű anyag megolvasztásához mennyi hő szükséges.
\end{itemize}
\end{af}
\begin{af}[Párolgás]
Olyan jelenség, melynek során a folyadék felszínén lévő nagyobb energiájú részecskék folyamatos energiaközlés hatására kiszakadnak a folyadékból és szabad állapotúvá válnak, azaz gáz halmazállapot jön létre. Ekkor a közölt hő a részecskék közötti kölcsönhatások megszüntetésére fordítódik.
\end{af}
\begin{af}[Párolgáshő]
\begin{itemize}
    \item %definíció:
    $L_p=\dfrac{Q}{m}$
    \item $\left[L_p\right]=\dfrac{J}{kg}$
    \item Megmutatja az egységnyi tömegű folyadék változatlan hőmérsékleten történő elpárologtatásához mennyi hő szükséges.
\end{itemize}
\end{af}
\begin{af}[Forrás]
A forrás olyan jelenség, melynek során a folyadék belsejében lévő alacsonyabb energiájú részecskék folyamatos energiaközlés hatására egy állandósult hőmérsékleten (forráspont) társaiktól elszakadva szabad állapotúvá válnak, azaz, gáz halmazállapot jön létre. Ekkor a közölt hő a részecskék közötti kölcsönhatások megszüntetésére fordítódik.
\end{af}
\begin{af}[Forráshő]
\begin{itemize}
    \item %definíció:
    $L_f=\dfrac{Q}{m}$
    \item $\left[L_f\right]=\dfrac{J}{kg}$
    \item Megmutatja az egységnyi tömegű, előzőleg forráspontjára felmelegített folyadék elforralásához szükséges hőközlés mértékét.
\end{itemize}
\end{af}



\newpage
\section{Egyenáram, mágneses mező}
\begin{af}[Elektromos áram]
Töltéshordozók rendezett mozgása (áramlása).
\end{af}
\begin{af}[Áramerősség]
\begin{itemize}
    \item %definíció: 
    $I=\dfrac{Q}{\Delta t}$
    \item %mértékegysége:
    $\left[I\right]=\dfrac{C}{s}=A$
    \item Megmutatja az egységnyi idő alatt a vezető teljes keresztmetszetén átáramló töltés mennyiségét.
    \item Iránya a pozitív töltéshordozók mozgásának iránya.
\end{itemize}
\end{af}
\begin{af}[Fogyasztó]
Az az eszköz vagy berendezés, amelyben elektromos áram hatására céljainknak megfelelő változások jönnek létre.
\end{af}
\begin{af}[Ellenállás]
\begin{itemize}
    \item %definíció:
    $R=\dfrac{U}{I}$
    \item %mértékegysége:
    $\left[R\right]=\dfrac{V}{A}=\Omega$
    \item Megmutatja az egységnyi áramerősség kialakításához szükséges feszültséget.
\end{itemize}
\end{af}
\begin{af}[Fajlagos ellenállás]
\begin{itemize}
    \item %definíció:
    $\varrho=\dfrac{RA}{l}$
    \item %mértékegysége$_1$:
    $\left[\varrho\right]=\dfrac{\Omega m^2}{m}=\Omega m$
    \item %mértékegysége$_2$:
    $\left[\varrho\right]=\dfrac{\Omega mm^2}{m}=\Omega m\cdot10^{-6}$
    \item Megmutatja az egységnyi hosszúságú és keresztmetszetű vezető ellenállását.
\end{itemize}
\end{af}
\begin{af}[Eredő ellenállás]
Olyan helyettesítő ellenállás, amelyre ugyanakkora feszültséget kapcsolva, ugyanakkora áram folyik, mint az eredeti áramkörben.
\end{af}
\begin{af}[Soros kapcsolás]
\begin{enumerate}
    \item[a)] $I_1=I_2=I_3$ \\
    Az áramerősség az áramkör minden pontján ugyanakkora.
    \item[b)] $U_1+U_2=U$ \\
    Az egyes áramköri elemekre eső feszültségek összege megegyezik az áramforrás feszültségével.
    \item[c)] ${R_e}=\sum{{R_i}}$ \\
    Az eredő ellenállás az egyes ellenállások összegeként adható meg.
\end{enumerate}
\end{af}
\begin{af}[Párhuzamos kapcsolás]
\begin{enumerate}
    \item[a)] $U_1=U_2=U$ \\
    Az egyes áramköri elemeken mérhető feszültség megegyezik az áramforrás feszültségével.
    \item[b)] $I=I_1+I_2$ \\
    A mellékágakban folyó áramerősségek összege megegyezik a főág áramerősségével.
    \item[c)] $\dfrac{1}{R_e}=\sum{\dfrac{1}{R_i}}$ \\
    Az eredő ellenállás reciproka megegyezik az egyes ellenállások reciprokának összegével.
\end{enumerate}
\end{af}
\begin{af}[Csomópont]
Az áramkör azon pontjai, ahol legalább 3 vezető fut össze.
\end{af}
\begin{af}[Kirchhoff I. törvénye]
Egy csomópontban az áramerősségek algebrai összege zérus.
$$\sum_{cs}I_i=0$$
\end{af}
\begin{af}[Hurok]
Ágak önmagukba visszafutó láncolata.
\end{af}
\begin{af}[Kirchhoff II. törvénye]
Egy irányított hurokra az ohmikus feszültségesések és az áramforrások feszültségének előjeles összege zérus.
$$\sum_h{I_iR_i}+\sum_hU_{0i}=0$$
\end{af}



\newpage
\section{Magnetosztatika}
\begin{af}[Mágneses tér]
Áramvezető által keltett, áramvezetőre ható tér.
\end{af}
\begin{af}[Mágneses indukció]
\begin{itemize}
    \item %definíció: 
    $B=\dfrac{M_{max}}{NIA}$
    \item %mértékegysége$_1$: 
    $\left[B\right]=\dfrac{Nm}{Am^2}=\dfrac{N}{Am}=T$
    \item %mértékegysége$_2$: 
    $\left[B\right]=\dfrac{Nm}{Am^2}=\dfrac{J}{Am^2}=\dfrac{VAs}{Am^2}=\dfrac{Vs}{m^2}=T$
    \item Megmutatja az egységnyi mágneses nyomatékú magnetométerre ható maximális forgatónyomatékot.
    \item Iránya az egyensúlyi helyzetben lévő magnetométer pozitív normálisának iránya.
\end{itemize}
\end{af}
\begin{af}[Mágneses indukcióvonalak minőségi jelentése]
Olyan görbék, melyek bármely pontjába húzott érintő az ottani $\vec{B}$ vektor tartóegyenese.
\end{af}
\begin{af}[Mágneses indukcióvonalak mennyiségi jelentése]
Az indukcióvonal-képet úgy kell megrajzolni, hogy ahol az indukció $B$ nagyságú, ott a merőlegesen felvett $A_\bot$ felületen $\Phi=BA_\bot$ számú indukcióvonal haladjon keresztül.
\end{af}
\begin{af}[Forráserősség]
$A$ zárt felületre összegezzük az $EA_\bot$ szorzatokat. (zárt felület teljes fluxusa)
$$N_E=\Psi{ö}=\sum_{A}{EA}_\bot$$
\end{af}
\begin{af}[Maxwell I. törvénye]
A $V$ térfogat forráserőssége megegyezik a térfogatba zárt töltések algebrai összegének $\dfrac{1}{\epsilon_0}$-szorosával.
$$N_E=\dfrac{1}{\epsilon_0}\sum Q_i$$
\end{af}
\begin{af}[Örvényerősség]
Egy tetszőleges irányított $g$ zárt görbe által határolt felület örvényerőssége az $\vec{E}$ és $\Delta\vec{s}$ vektorok skaláris szorzatának összegével egyezik meg.
\[Ö_E=\sum_{g}^{\circ} \vec{E}\Delta\vec{s}\]
\end{af}
\begin{af}[Maxwell II. törvénye]
Az elektrosztatikus tér örvénymentes.
$$Ö_{E}=0$$
\end{af}
\begin{af}[Maxwell III. törvénye]
A magnetosztatikus tér forrásmentes.
$$N_{B}=0$$
\end{af}
\begin{af}[Maxwell IV. törvénye (Amper-féle gerjesztési törvény)]
Egy tetszőleges irányított $g$ zárt görbe által körülhatárolt felület mágneses áramerőssége egyenesen arányos a feületet átdöfő áramok erősségének algebrai összegével.
$$Ö_{B}=\mu_0\sum I_i$$
\end{af}
\begin{af}[Lorentz-erő]
Homogén, $B$ indukciójú mágneses mezőben az indukcióvonalakra merőlegesen elhelyezett $l$ hosszúságú $I$ árammal átjárt vezetőre ható erő nagysága $F=B\cdot I\cdot l$
\begin{itemize}
    \item Áramvezetőre ható Lorentz-erő: $\vec{F_L}=I[\vec{l}\times\vec{B}]$
    \item Mozgó töltéshordozóra ható Lorentz-erő: $\vec{F_L}=q[\vec{v}\times\vec{B}]$
\end{itemize}
\end{af}
\begin{af}[Abszolút amper]
Akkor $1\ A$ az áram erőssége egy vékony egyenes vezetőben, ha a vákuumban tőle $1\ m$ távolságra elhelyezett ugyanekkore árammal átjárt egyenes vezető $1\ m$-es darabjára $2\cdot10^{-7}\ N$ erő hat.
\end{af}
\begin{af}[Lenz törvénye]
Az indukált áram iránya mindig olyan, hogy az őt létrehozó hatást csökkenteni igyekszik.
\end{af}
\begin{af}[Mozgási indukció]
Mozgási indukcióról beszélünk, ha (időben) állandó mágneses mezőben a váltakozó felületű vezetőhurokban indukálódik a feszültség.]
\end{af}
\begin{af}[Effektív áram erőssége]
Annak az egyenáramnak az erőssége, amely ugyanabban a vezetőben ugyanannyi idő alatt ugyanannyi hőt termel.
\end{af}\\ \\
\centerline{Lektorálta: Monori J. Bence}



\newpage
\section{Harmonikus rezgőmozgás}
\begin{af}[Amplitúdó]
Az egyensúlyi helyzettől mért maximális szélső távolság.
\end{af}
\begin{af}[Referencia körmozgás]
Az egyenletes körmozgást végző tömegpontnak a kör síkjában lévő egyenesre eső vetülete harmonikus
rezgőmozgást végez.
\end{af}
\begin{af}[Harmonikus erő]
Harmonikus erőről beszélünk, ha az erő nagysága egyenesen arányos a kitéréssel, és iránya azzal ellentétes.
\end{af}
\begin{af}[Csillapodó rezgőmozgás]
Csillapodó rezgésről beszélünk, ha a rezgő rendszerben disszipatív kölcsönhatások is fellépnek, aminek hatására az amplitúdó csökken.
\end{af}
\begin{af}[Szabadrezgés]
Szabadrezgést végez az az adott rezgési energiával rendelkező rendszer, amelyet magára hagyunk és a saját paraméterei ($D$ és $m$) által meghatározott frekvenciával rezeg.
$$ \omega_0 = \sqrt{\dfrac{D}{m}}
\end{af}
\begin{af}[Kényszerrezgés]
Kényszerrezgésről beszélünk, ha egy rezgő rendszerre kívülről hat egy periodikusan változó külső erő.
\end{af}
\begin{af}[Rezonancia]
Rezonanciáról beszélünk, ha a gerjesztő rezgés körfrekvenciája megegyezik a gerjesztett rezgés körfrekvenciájával.
\end{af}
\begin{af}[Rezonanciagörbe]
A gerjesztett rezgés amplitúdója akkor maximális, ha rezonancia lép fel.
\end{af}



\newpage
\section{Hullámtan}
\begin{af}[Azonos fázisú pontok]
$\Delta\varphi=2k\pi$
\end{af}
\begin{af}[Haladó hullám]
Haladó hullámról beszélünk, ha egy rugalmas közegben a rezgés fázisa és vele együtt a rezgési energia terjed.
\end{af}
\begin{af}[Hullámhossz]
Az azonos fázisú pontok közötti legrövidebb távolság. Jele: $\lambda$
\end{af}
\begin{af}[Transzverzális hullám]
Transzverzális hullámról beszélünk, ha a részecskék rezgésiránya merőleges a hullám terjedési irányára.
\end{af}
\begin{af}[Longitudinális hullám]
Longitudinális hullámról beszélünk, ha a részecskék rezgésiránya mergegyezik a hullám terjedési irányával.
\end{af}
\begin{af}[Hullámfelület]
Az azonos fázisú pontokat összekötő vonalak.
\end{af}
\begin{af}[Hullámfront]
Az (elől haladó) $ \pi / 2 $ fázisú pontok összessége.
\end{af}
\begin{af}[Hullámtér]
A térnek azon része, ahova a hullám már eljutott.
\end{af}
\begin{af}[Sugár(irány)]
A hullámfelületre merőleges irány.
\end{af}
\begin{af}[Huygens-elv]
A hullámfelület minden pontja úgynevezett elemi hullámok kiindulópontja és a későbbi hullámtérbeli hatást ezen elemi hullámok burkolófelülete adja.
\end{af}
\begin{af}[Huygens-Fresnel-elv]
A hullámfelület minden pontja úgynevezett elemi hullámok kiindulópontja és a későbbi hullámtérbeli hatást ezen elemi hullámok interferenciája adja meg.
\end{af}
\begin{af}[Snellius-Descartes-törvény]
A beesési szög ($\alpha$) szinuszának és a törési szög ($\beta$) szinuszának aránya a közegekben mért terjedési sebességek arányával egyenlő, ami a két közeg relatív törésmutatója.
$$ \dfrac{sin\alpha}{\sin\beta} = \dfrac{c_1}{c_2} = n_{2,1} $$
\end{af}
\begin{af}[Interferencia]
Időben állandósuló hullám szuperpozíció.
Feltétele a koherencia.
\end{af}
\begin{af}[Erősítés feltétele]
\begin{itemize}
    \item 1 dimenzióban: $ A = A_1 + A_2 $ és $\Delta\varphi = 2k\pi $
    \item 2 dimenzióban: előzőek és $ \Delta r = \lvert r_1+r_2 \rvert = 2k\pi $
\end{itemize}
\end{af}
\begin{af}[Gyengítés feltétele]
\begin{itemize}
    \item 1 dimenzióban: $ A = \lvert A_1 - A_2 \rvert $ ($A_1=A_2 $ esetén $A=0$ a kioltás) és $\Delta\varphi = (2k+1)\pi $
    \item 2 dimenzióban: előzőek és $ \Delta r = \lvert r_2-r_1 \rvert = (2k+1)\dfrac{\pi}{2} $
\end{itemize}
\end{af}
\begin{af}[Koherencia]
A találkozó hullámok fáziskülönbsége állandó.
\end{af}
\begin{af}[Állóhullám]
A részecskék  rezgés amplitúdója különböző, de időben állandó és a fázis nem terjed.
\end{af}

\par $\;$

\end{document}
